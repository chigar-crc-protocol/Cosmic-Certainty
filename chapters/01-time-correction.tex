\chapter{تصحيح الزمن والنسبة الكونية | Time Correction}

\section{قانون الحركة الكونية}
\bilingual{
نعتمد في هذا البحث على قانون الحركة (المؤثر والنتيجة)، حيث أن العجز التراكمي في ساعات شروق القمر (10 ساعات) يتطابق مع الثابت الكوني الموحد.
}{
In this research, we rely on the law of motion (cause and effect), where the cumulative deficit in moonrise hours (10 hours) aligns perfectly with the cosmic constant.
}

\begin{sovereignbox}{الثابت السيادي للمجال}
\[ \text{Constant} = 800 \times 43 = 34,400 \]
\end{sovereignbox}

\bilingual{
هذا الرقم ليس مجرد ناتج حسابي، بل هو الضابط الهندسي لمحيط الأرض ومسافة الشمس، وهو ما يثبت مركزية مكة في النظام الهندسي الموحد.
}{
This number is not merely a calculation; it is the geometric regulator for Earth's circumference and the Sun's distance, proving Mecca's centrality in the Unified Geometric System.
}
