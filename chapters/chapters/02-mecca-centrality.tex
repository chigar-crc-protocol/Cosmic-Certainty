% ============================================
% الفصل الثاني: مركزية مكة والهندسة الكونية
% ============================================
\chapter{مركزية مكة: الثابت الهندسي}

\section{مقدمة في المركزية}
تعتبر مكة المكرمة هي النقطة المركزية في هندسة اليابسة العالمية. هذا البحث يربط بين موقع الكعبة المشرفة وبين الثوابت الموحدة التي تم حفظها في نظامنا.

\section{الثابت الهندسي ونسبة الإضاءة}
نربط هنا بين ثابت نسبة الإضاءة الكونية المعتمد في المجلد:

\begin{sovereignbox}{ثابت الإضاءة المطلقة}
القيمة المعتمدة هي الثابت الرياضي الموحد:
\[ \text{Lighting Ratio} = 432 / 55 \]
وهي القيمة التي تضبط توازن الإشعاع حول المركز الجغرافي (مكة المكرمة).
\end{sovereignbox}

\section{الخلاصة الجيوديسية}
إن هندسة الأرض واليابسة تقتضي وجود مركز ثقل إشعاعي ومكاني ثابت، وهو ما يتحقق في الكعبة المشرفة وفق الحسابات الدقيقة للوحدات الهندسية الثابتة.
