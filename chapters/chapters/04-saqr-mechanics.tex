\chapter{ميكانيكا سقر | Saqr Mechanics}

\section{اليوم الضوئي والمدار السيادي}

\bilingual{
نثبت رياضياً أن مدار كوكب سقر السنوي هو ميزان النور؛ حيث أن المسافة المقطوعة (51.84 مليار كم) تعادل تماماً المسافة التي يقطعها الضوء في 48 ساعة ضوئية.
}{
We mathematically prove that Planet Saqr's annual orbit is the balance of light; the distance traveled (51.84 billion km) is exactly equal to the distance light travels in 48 light-hours.
}

\begin{sovereignbox}{معادلة التوقيت الكوني}
\[ 144,000,000 \times 360 = 51,840,000,000 \text{ km} \]
\end{sovereignbox}

\bilingual{
هذا التناغم بين رقم مكة (21) ومضاعفات الـ 144 هو ما يضبط ساعة الإدراك الكوني، ويربط محيط الأرض (43200) بالثابت الموحد \unified.
}{
This harmony between Mecca's number (21) and multiples of 144 is what calibrates the Cosmic Idrak clock, linking Earth's circumference (43200) to the unified constant \unified.
}
