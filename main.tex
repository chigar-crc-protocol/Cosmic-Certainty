\documentclass[12pt,a4paper,oneside]{book}

% --- 1. الإعدادات واللغة ---
\usepackage[top=2.8cm,bottom=2.8cm,left=2.5cm,right=2.5cm]{geometry}
\usepackage{fontspec}
\usepackage{polyglossia}
\usepackage{amsmath,amssymb}
\usepackage{tikz}
\usepackage{tcolorbox}
\setmainlanguage{arabic}
\newfontfamily\arabicfont[Script=Arabic,Scale=1.2]{Amiri}
\definecolor{SovereignBlue}{RGB}{0,33,71}

\begin{document}

% --- 2. الغلاف السيادي ---
\begin{titlepage}
\centering
\vspace*{1.5cm}
\begin{tcolorbox}[colframe=SovereignBlue, colback=white, arc=20pt, linewidth=5pt, center, width=0.9\textwidth]
    \centering
    \vspace{0.8cm}
    {\Huge\bfseries المجلد الكوني اليقين}\\[0.5cm]
    {\Large\bfseries مصفوفة الإدراك الهندسي والزمني}\\[0.3cm]
    {\small الإصدار السيادي الموحد — 2026}
    \vspace{0.8cm}
\end{tcolorbox}

\vspace{2cm}
\begin{tikzpicture}
    \draw[line width=5pt, SovereignBlue] (0,0) circle (3.5);
    \node at (0,0.3) {\fontsize{50}{55}\selectfont\bfseries 34400};
    \node at (0,-1.2) {\Large\bfseries القيمة السيادية الموحدة};
\end{tikzpicture}

\vfill
{\Large\bfseries الباحث: هشام شيكر}\\[0.4cm]
{\large المرجعية العلمية: بيانات الإمام المهدي ناصر محمد اليماني}\\[0.8cm]
\hrule
\vspace{0.4cm}
{\small تم التوثيق الرقمي في: 13 فبراير 2026 م}
\end{titlepage}

% --- 3. الفصل الأول ---
\chapter{مركزية بكة ومصفوفة الإدراك}
إن هذا المجلد يوثق الحقيقة الهندسية المطلقة لمركزية مكة المكرمة (بكة) كقلب نابض لليابسة، وتناغم أحداث الكون مع الثابت الموحد 34400.

\end{document}
