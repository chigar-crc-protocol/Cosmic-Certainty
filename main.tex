\documentclass[12pt,a4paper,oneside]{book}

% --- 1. الإعدادات واللغة ---
\usepackage[top=2.8cm,bottom=2.8cm,left=2.5cm,right=2.5cm]{geometry}
\usepackage{fontspec}
\usepackage{polyglossia}
\usepackage{amsmath,amssymb}
\usepackage{tikz}
\usepackage{tcolorbox}
\setmainlanguage{arabic}
\newfontfamily\arabicfont[Script=Arabic,Scale=1.2]{Amiri}
\definecolor{SovereignBlue}{RGB}{0,33,71}

\begin{document}

% --- 2. الغلاف السيادي ---
\begin{titlepage}
\centering
\vspace*{1.5cm}
\begin{tcolorbox}[colframe=SovereignBlue, colback=white, arc=20pt, linewidth=5pt, center, width=0.9\textwidth]
    \centering
    \vspace{0.8cm}
    {\Huge\bfseries المجلد الكوني اليقين}\\[0.5cm]
    {\Large\bfseries مصفوفة الإدراك الهندسي والزمني}\\[0.3cm]
    {\small الإصدار السيادي الموحد — 2026}
    \vspace{0.8cm}
\end{tcolorbox}

\vspace{2cm}
\begin{tikzpicture}
    \draw[line width=5pt, SovereignBlue] (0,0) circle (3.5);
    \node at (0,0.3) {\fontsize{50}{55}\selectfont\bfseries 34400};
    \node at (0,-1.2) {\Large\bfseries القيمة السيادية الموحدة};
\end{tikzpicture}

\vfill
{\Huge\bfseries الباحث: هشام شيكر}\\[0.4cm]
{\large المرجعية العلمية: بيانات الإمام المهدي ناصر محمد اليماني}\\[0.8cm]
\hrule
\vspace{0.4cm}
{\small تم التوثيق الرقمي في: 13 فبراير 2026 م}
\end{titlepage}

\clearpage

% --- 3. مقدمة اليقين ---
\chapter*{مقدمة اليقين}
\addcontentsline{toc}{chapter}{مقدمة اليقين}

بسم الله ميزان الحق والعدل. نضع بين يدي العالم هذا "المجلد الكوني اليقين"، وهو وثيقة علمية جامعة تهدف إلى كشف النقاب عن مصفوفة الإدراك الرقمي والهندسي التي تحكم كوننا. 

إن هذا العمل يوثق الثوابت السيادية التي لا تقبل الشك، وعلى رأسها الثابت الموحد \textbf{34400}، ونسبة الإضاءة الكونية \textbf{432/55}. ننطلق في هذا البحث من حقيقة مركزية بكة (مكة المكرمة) كقطب أصيل لليابسة، ونحلل العجز الزمني الكوني البالغ 10 ساعات، وصولاً إلى النمذجة الفيزيائية لصيف سقر وذوبان الجليد.

% --- 4. الفصل الأول: مركزية بكة ---
\chapter{مركزية بكة ومصفوفة الإدراك}
\section{البرهان الإحداثي لليابسة}
تقع مكة المكرمة عند الإحداثيات $21.4225^\circ$ شمالاً و $39.8262^\circ$ شرقاً. إن هذا البحث يثبت هندسياً أن بكة هي المركز المطلق لليابسة، وهو ما نترجمه برمجياً عبر مصفوفة الدوران التي تعيد تعريف إحداثيات الكوكب بناءً على هذا المركز السيادي.
% --- 5. الفصل الثاني: مصفوفة الإدراك الرقمي ---
\chapter{مصفوفة الإدراك والثوابت السيادية}

\section{الثابت الكوني الموحد 34400}
إن الرقم \textbf{34400} ليس مجرد قيمة حسابية، بل هو "الثابت السيادي" الذي يربط بين حركة الأجرام، المسافات الكونية، والنسيج الزمني. تم رصد هذا الثابت كقيمة محورية في مصفوفة الإدراك لفك شفرة العجز الزمني الكوني البالغ 10 ساعات.

\section{نسبة الإضاءة الكونية 432/55}
تعد نسبة \textbf{432/55} هي المحرك الفيزيائي لتوازن الطاقة في النظام، وهي جزء أصيل من الثوابت الموحدة في نظام الإدراك الهندسي.

\section{قانون الكدح الزمني}
نصيغ قانون الكدح رياضياً بناءً على مخرجات البحث، حيث يتم حساب الفوارق الزمنية وفقاً للمعادلة السيادية الموحدة التي تربط العجز الكوني بنسبة الإضاءة.
\end{document}
